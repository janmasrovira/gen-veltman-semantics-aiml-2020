\documentclass{beamer}
% \usetheme{Luebeck}
% \usetheme{Madrid}
\usetheme{Boadilla}

\usepackage{unicode-math}
\usepackage{amsmath}
\usepackage{amssymb}
\usepackage{mathtools}
\usepackage{latexsym}
\usepackage{cancel}
\usepackage{xcolor}
\usepackage{xspace}


\title{Generalised Veltman Semantics in Agda}
\author{Joost J. Joosten\inst{1} \and Jan Mas Rovira\inst{1}\and Luka Mikec\inst{2}}
\institute[shortinst]{\inst{1} University of Barcelona \and %
  \inst{2} University of Zagreb}
\date{August 2020}

% Luka: Hi all! I agree with Joost's suggestion. Jan, also take into account
% that since our talk is a short presentation we have about 10 minutes to
% present (and then 5 minutes for questions). I'd suggest to focus on basics
% (what is Veltman semantics, what is Agda etc.) and less on the results

% Joost: Hi Jan, when composing the slides, please be very modest in your aims
% Here's what Luka writes about his presentation
% I'd suggest we focus on interpretability logics in general, then a bit about IL(All),
% Indeed, it is good to get the basics straight
% So, the first --at least-- five minutes should be dedicated entirely to
% a slow and gentle introduction to interpretability logics
% if not, the first seven
% if you want, you can sketch the idea
% but no full proofs
% ideas and summaries are more important

\newcommand{\gl}{\ensuremath{\textup{\textbf{GL}}}\xspace}


\begin{document}

\frame{\titlepage}

\begin{frame}
  \frametitle{Introduction}
  \begin{itemize}
  \item Generalised Veltman semantics are a kind of relational semantics (à la
    Kripke) for \textbf{interpretability logics}. \break \pause
    \item Agda is a \textbf{proof assistant} based on dependent type theory.
  \end{itemize}
\end{frame}

\begin{frame}
  \frametitle{Interpretability Logics}
  In the logic of provability \gl we have that:
  \begin{itemize}
    \item $□A$ means ``A is provable''.
    \item $♢A$ means ``A is consistent''. Note that $♢A=¬□¬A$.
  \end{itemize}
  The aim of \gl is to describe the provably structural behaviour of the
  formalised provability predicate.

  \pause \vspace{0.5cm}

  Interpretability logics extend the provability logic \gl. There is an
  additional binary modal operator $▷$.
  \begin{itemize}
  \item $A▷B$ means ``A interprets B''.
  \end{itemize}
  The aim is to find a logic that describes exactly all provably structural
  behaviour of the notion of interpretability.
\end{frame}

\begin{frame}
  \frametitle{Interpretability Logics}
  \textbf{Definition}
  An interpretation of a theory $A$ into a theory $B$ is a translation $⋆$ that
  maps the non-logical symbols in the language of $A$ to formulas in the
  language of $B$ with the same free variables, so that
  \[⊢_Aϕ⇒\ ⊢_Bϕ^⋆\]
  Which one is preferred?
  \[A⊢ϕ⇒ B⊢ϕ^⋆\]
\end{frame}


\end{document}