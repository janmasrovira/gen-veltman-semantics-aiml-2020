\documentclass{beamer}
% \usetheme{Luebeck}
% \usetheme{Madrid}
\usetheme{Boadilla}

\usepackage{unicode-math}
\usepackage{amsmath}
\usepackage{amssymb}
\usepackage{mathtools}
\usepackage{latexsym}
\usepackage{cancel}
\usepackage{xcolor}
\usepackage{xspace}
\usepackage{textcomp}



\title{Generalised Veltman Semantics in Agda}
\author{Joost J. Joosten\inst{1} \and Jan Mas Rovira\inst{1}\and Luka Mikec\inst{2}}
\institute[shortinst]{\inst{1} University of Barcelona \and %
  \inst{2} University of Zagreb}
\date{August 26th 2020}

% Luka: Hi all! I agree with Joost's suggestion. Jan, also take into account
% that since our talk is a short presentation we have about 10 minutes to
% present (and then 5 minutes for questions). I'd suggest to focus on basics
% (what is Veltman semantics, what is Agda etc.) and less on the results

% Joost: Hi Jan, when composing the slides, please be very modest in your aims
% Here's what Luka writes about his presentation
% I'd suggest we focus on interpretability logics in general, then a bit about IL(All),
% Indeed, it is good to get the basics straight
% So, the first --at least-- five minutes should be dedicated entirely to
% a slow and gentle introduction to interpretability logics
% if not, the first seven
% if you want, you can sketch the idea
% but no full proofs
% ideas and summaries are more important

% We cannot assume that people know the logics that we work with
% so we have to slowly introduce them and motivate them
% give some examples
% The last five minutes can be used to hint at the work we have done
% Or last four minutes
\newcommand{\prin}[1]{\ensuremath{\textup{#1}}\xspace}
\newcommand{\il}{\ensuremath{\textup{IL}}\xspace}
\newcommand{\ilm}{\ensuremath{\textup{ILM}}\xspace}
\newcommand{\gl}{\ensuremath{\textup{GL}}\xspace}
\newcommand{\zf}{\ensuremath{\textup{ZF}}\xspace}
\newcommand{\veql}{\ensuremath{\textup{(V=L)}}\xspace}


\begin{document}

\frame{\titlepage}

\begin{frame}
  \frametitle{Introduction}
  \begin{itemize}
  \item Generalised Veltman semantics are a kind of relational semantics (à la
    Kripke) for \textbf{interpretability logics}. \break \pause
    \item Agda is a \textbf{proof assistant} based on dependent type theory.
  \end{itemize}
\end{frame}

\begin{frame}
  \frametitle{Interpretability Logics}
  In the logic of provability \gl we have that:
  \begin{itemize}
    \item $□A$ means ``A is provable''.
    \item $♢A$ means ``A is consistent''. Note that $♢A=¬□¬A$.
  \end{itemize}
  The aim of \gl is to describe and prove the structural behaviour of the
  formalised provability predicate.

  \pause \vspace{0.5cm}

  Interpretability logics extend the provability logic \gl. There is an
  additional binary modal operator $▷$.
  \begin{itemize}
  \item $A▷B$ means ``A interprets B''.
  \end{itemize}
  The aim is to find a logic that describes exactly the behaviour of the notion
  of interpretability, which is yet to be discovered.
\end{frame}

\begin{frame}
  \frametitle{Interpretability Logics}
  \textbf{Definition} We say that a theory U interprets V if there is an
  interpretation of V into U.

  An interpretation of a theory $U$ into a theory $V$ is a translation $⋆$ that
  maps the non-logical symbols in the language of $U$ to formulas in the
  language of $V$ with the same free variables, so that
  \[U⊢ϕ⇒ V⊢ϕ^⋆\]
\end{frame}

\begin{frame}
  \frametitle{Why Interpretability Logics matter}
  Examples of interpretations (copied from Visser's overview):
  \vspace{0.4cm}
  \begin{itemize}
  \item The interpretation of arithmetic in set theory.
    \pause
  \item Gödel's interpretation of $\zf+\veql$ in \zf. This interpretation
    provides a relative consistency proof of \zf + CH w.r.t. \zf.
    \pause
  \item The interpretation of elementary syntax in arithmetic. Important for
    Gödel's incompleteness theorems.
  \end{itemize}

\end{frame}

\begin{frame}
  \frametitle{Logic \il}
  The logic \il is the base of all interpretability logics.

  \vspace{0.3cm}

  \il has all the theorems of \gl plus
  the following axioms schemes: \pause
  \begin{itemize}
  \item J1: $□ (A → B) → A ▷ B$;
    \pause
  \item J2: $A ▷ B ∧ B ▷ C → A ▷ C$;
    \pause
  \item J3: $(A ▷ C ∧ B ▷ C) → (A ∨ B) ▷ C$;
    \pause
  \item J4: $A ▷ B → ♢ A → ♢ B$;
    \pause
  \item J5: $♢ A ▷ A$.
  \end{itemize}
\end{frame}

\begin{frame}
  \frametitle{Logic \ilm}

  The logic ILM is defined as \il + \prin{M}, where
  \[M≔ A ▷ B → (A ∧ □ C) ▷ (B ∧ □ C).\]

  The theorems of $\textsf{ILM}$ are the set of interpretability principles that
  are always provable in theories which are $Σ_1$ sound, have full induction and
  prove consistency of its finite subsystems. An example of such theory
  is $\textsf{PA}$.

  % \vspace{0.4cm}
  % \pause
  % If $ILM⊢A▷B$ then we know that $?$.

\end{frame}


\begin{frame}
  \frametitle{Agda publications}
  Papers that use Agda (\textasciitilde 200):

  \url{https://wiki.portal.chalmers.se/agda/Main/PapersUsingAgda}

  \vspace{0.3cm}

  \url{https://researchr.org/bibliography/agda-papers/publications}
\end{frame}

\begin{frame}
  \centering \Huge Thank you!
\end{frame}


\end{document}